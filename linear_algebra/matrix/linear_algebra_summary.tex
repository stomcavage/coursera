\documentclass[11pt, oneside]{article}   	% use "amsart" instead of "article" for AMSLaTeX format
\usepackage{geometry}                		% See geometry.pdf to learn the layout options. There are lots.
\geometry{letterpaper}                   		% ... or a4paper or a5paper or ... 
%\geometry{landscape}                		% Activate for for rotated page geometry
%\usepackage[parfill]{parskip}    		% Activate to begin paragraphs with an empty line rather than an indent
\usepackage{graphicx}				% Use pdf, png, jpg, or eps§ with pdflatex; use eps in DVI mode
								% TeX will automatically convert eps --> pdf in pdflatex		
\usepackage{amssymb}
\usepackage{amsmath}

\title{Linear Algebra Summary}
\author{Steven Tomcavage}
%\date{}							% Activate to display a given date or no date

\begin{document}
\maketitle

\section{Vectors}

The set of all $\textsl{D}$ vectors over $\mathbb{R}$ is written $\mathbb{R}^{\textsl{D}}$, where $\mathbb{R}^{\textsl{D}} = \begin{bmatrix}x_\textsl{1}, x_\textsl{2}, \dots x_{D} \end{bmatrix}$. For example, $ \mathbb{R}^4 =\begin{bmatrix} x_1, x_2, x_3, x_4 \end{bmatrix} $. 

A vector with values that are zero is called a \textit{sparse} vector. If no more than \textit{k} entries are zero, then the vector is \textit{k-sparse}. If all the values in a vector are zero, then that vector is called a \textit{zero vector}.

\subsection{Vector addition}

Vector addition has the effect of translating the vector. That is, if the vector represents a line or a plane, then adding a vector to it will move the line or the place in space. Vector addition is calculated by summing the individual vector entries:

\begin{displaymath}
\begin{bmatrix}u_1, u_2, \dots, u_n\end{bmatrix} + \begin{bmatrix}v_1, v_2, \dots, v_n\end{bmatrix} = \begin{bmatrix}u_1 + v_1, u_2 + v_2, \dots, u_n + v_n \end{bmatrix}
\end{displaymath}

Vector addition is associative and commutative:

\begin{displaymath}
(x + y) + z = x + (y + z)
\end{displaymath}

\begin{displaymath}
x + y = y + x
\end{displaymath}

\subsection{Vector scalar multiplication}

Vector scalar multiplication has the effect of scaling the vector. That is, if the vector represents a line or plane, then multiplying it by a scalar increases or decreases the size of the line or plane. Vector scalar multiplication is defined as:
 \begin{displaymath}
\alpha \begin{bmatrix} v_1, v_2, \dots, v_n \end{bmatrix} = \begin{bmatrix} \alpha v_1, \alpha v_2, \dots \alpha v_n \end{bmatrix}
\end{displaymath}

Vector scalar multiplication is distributive over vector addition:

\begin{displaymath}
\alpha(u + v) = \alpha u + \alpha v
\end{displaymath}

An expression of the form $\alpha u + \beta v$ where $\alpha + \beta = 1$ is called an \textit{affine combination}. The set of all affine combinations of $u$ and $v$ is the line that passes through $u$ and $v$.

\subsection{Dot product}

The input to the dot product is two vectors and the output is a scalar. The dot product is sometimes called the \textit{scalar product}. The dot product of two vectors is defined as:

\begin{displaymath}
u \dot v = \Sigma u[k] v[k]
\end{displaymath}

\subsection{Linear combination}

A linear combination of vectors is defined as the sum of a series of scaled vectors:

\begin{displaymath}
\alpha_1 v_1 + \dots + \alpha_n v_n
\end{displaymath}

\subsection{Vector span}

A vector span is defined as the set of all linear combinations of some vectors $v_1, \dots, v_n$. It is written as $\text{Span} \{v_1, \dots, v_n \}$. A span of a single vectors is defined as $\text{Span} \{v\} = \{ \alpha v : a \in \mathbb{R}\}$.

\subsection{Generators}

A generator is defined as a set of vectors in \textsl{V}, $v_1, \dots, v_n$ that comprise $\text{Span}\{\textsl{V}\}$. 

\subsection{Abstract Vector Space}

An abstract vector space is a set of vectors \textsl{V} over a field \mathbb{F} that has an addition operation, a multiplication operation, a zero vector, and where the following holds true:

If \textsl{V} contains $v$, then it contains $\alpha v$ for every scalar $\alpha$

If \textsl{V} contains $u$ and $v$, then it contains $u + v$.

\subsection{Affine space and affine combination}

An affine space for the vector c and the vector space \textsl{V} is defined as $c + \textsl{V}$. 

An affine combination over the vectors $u_1, u_2, u_3$ is $\gamma u_1 + \alpha u_2 + \beta u_3$, where $\gamma + \alpha + \beta = 1$. The set of all affine combinations over $u_1, u_2, u_3$ is called the affine hull.

\subsection{Homogenious linear system}

A linear equation in the form $a \dot x = 0$ is a homogenious linear equation. A set of homogenious linear equations is a homogenious linear system. The solution set for a homogenious linear system is either empty or an affine space.

\section{Matrix}

\subsection{Matrix vector multiplication}

\subsection{Vector matrix multiplication}

\subsection{Matrix Matrix multiplication}

\end{document}  